%%%%%%%%%%%%%%%%%%%%%%%%%%%%%%%%%%%%%%%%%%%%%%%%%%%%%%%%%%%%%%%%%%%%%%%%
\chapter{System Design and Specification}
%%%%%%%%%%%%%%%%%%%%%%%%%%%%%%%%%%%%%%%%%%%%%%%%%%%%%%%%%%%%%%%%%%%%%%%%

%%%%%%%%%%%%%%%%%%%%%%%%%%%%%%%%%%%%%%%%%%%%%%%%%%%%%%%%%%%%%%%%%%%%%%%%
\section{Requirements Analysis}
%%%%%%%%%%%%%%%%%%%%%%%%%%%%%%%%%%%%%%%%%%%%%%%%%%%%%%%%%%%%%%%%%%%%%%%%

The goal of this section is to identify and define the needs and expectations of the users of the Haskify. Haskify is a web-based educational tool designed to support students learning functional programming through Haskell. It integrates an AI assistant, a Haskell code editor with a live compiler.
\\ 
\newline
\textbf{Primary Users:}
\begin{itemize}
  \item \textbf{Students:} Interact with the system to write code, receive feedback, and learn.
  \item \textbf{Educators:} Could review usage or adapt exercises in future extensions.
\end{itemize}
\\
\textbf{System Objectives:}
\begin{itemize}
  \item Provide a web-based Haskell code editor with syntax highlighting and compilation.
  \item Deliver AI-driven guidance and explanation for functional programming exercises.
  \item Enhance the learning experience through immediate feedback.
\end{itemize}

%%%%%%%%%%%%%%%%%%%%%%%%%%%%%%%%%%%%%%%%%%%%%%%%%%%%%%%%%%%%%%%%%%%%%%%%
\section{Functional Requirements}
%%%%%%%%%%%%%%%%%%%%%%%%%%%%%%%%%%%%%%%%%%%%%%%%%%%%%%%%%%%%%%%%%%%%%%%%

\begin{enumerate}[label=FR\arabic*:]
  \item The system shall allow users to write and edit Haskell code in the code editor.
  \item The system shall compile and execute Haskell code and display the output or errors.
  \item The system shall provide an AI assistant that accepts user questions and code-related queries.
  \item The AI assistant shall respond with educational hints or explanations related to Haskell and functional programming.
  \item The system shall detect and highlight syntax or runtime errors in user code.
\end{enumerate}

%%%%%%%%%%%%%%%%%%%%%%%%%%%%%%%%%%%%%%%%%%%%%%%%%%%%%%%%%%%%%%%%%%%%%%%%
\section{Non-Functional Requirements}
%%%%%%%%%%%%%%%%%%%%%%%%%%%%%%%%%%%%%%%%%%%%%%%%%%%%%%%%%%%%%%%%%%%%%%%%

\begin{enumerate}[label=NFR\arabic*:]
  \item \textbf{Usability:} The interface should be intuitive and accessible to first-year students.
  \item \textbf{Performance:} The code compilation and AI responses must complete within 2 seconds under normal load.
  \item \textbf{Availability:} The system should be available 24/7 with minimal downtime.
  \item \textbf{Scalability:} The system should support at least 100 concurrent users during peak times.
  \item \textbf{Maintainability:} The codebase should be modular and follow good software engineering practices.
  \item \textbf{Portability:} The app must run in all major browsers (Chrome, Firefox, Safari).
\end{enumerate}
These requirements were considered throughout the development process. For example, usability was prioritized by selecting Monaco Editor, which provides students with a familiar coding environment similar to Visual Studio Code. Performance concerns were addressed by limiting AI context and optimizing the Haskell interpreter's execution environment. We avoided complex UI frameworks to reduce load times and improve responsiveness.
To meet availability and scalability goals, the system was designed using stateless APIs and could be deployed on scalable platforms like Vercel or AWS. Portability was ensured by rigorous testing across modern browsers. These constraints directly influenced design choices and technology selection.

%%%%%%%%%%%%%%%%%%%%%%%%%%%%%%%%%%%%%%%%%%%%%%%%%%%%%%%%%%%%%%%%%%%%%%%%
\newpage
\section{Use Case Diagram}
%%%%%%%%%%%%%%%%%%%%%%%%%%%%%%%%%%%%%%%%%%%%%%%%%%%%%%%%%%%%%%%%%%%%%%%%

Below is the use case diagram for Haskify

\begin{figure}[h!]
\centering
\begin{tikzpicture}

% Use-Case-System begin
\begin{umlsystem} [x=0, y=0, fill=pink!20] {{Haskify}}
\umlusecase[name=case1, width=2cm, x=1]{Write Haskell Code}
\umlusecase[name=case2, width=2cm, y=-7] {Run Code} 
\umlusecase[name=case3, width=2cm, y=-3.75] {Chat with AI Assistant} 
\umlusecase[name=case4, width=2cm, x=7, y=-3] {View Output / Errors} 
\umlusecase[name=case5, width=2.5cm, x=7, y=-6] {Receive Explanation / Feedback} 
\umlusecase[name=case6, width=2.5cm, x=7, y=-9] {return Explanation / Feedback} 
\umlusecase[name=case7, width=2cm, y=-9.5] {Return Output / Errors} 
\end{umlsystem}
% end Use-Case-System 
%Actors
\umlactor[x=-4, y=0]{Student}
\umlactor[x=-4, y=-4.5]{AI Assistant}
\umlactor[x=-4, y=-8]{Compiler Service}
%Relations
\umlassoc{Student}{case1}
\umlassoc{Student}{case2}
\umlassoc{Student}{case3}
\umlassoc{Student}{case4}
\umlassoc{Student}{case5}
\umlassoc{AI Assistant}{case6}
\umlassoc{Compiler Service}{case7}
\umlinclude{case2}{case7}
\umlinclude{case3}{case5}
\end{tikzpicture}
\end{figure}
\subsection*{\LARGE Use Case Specifications}
\textbf{Use Case: Write Haskell Code}
\begin{itemize}
  \item \textbf{Actor:} Student
  \item \textbf{Description:} The student writes Haskell code in the integrated editor.
  \item \textbf{Preconditions:} The user has opened the editor interface.
  \item \textbf{Postconditions:} The code is stored in memory and ready to be compiled.
\end{itemize}

\vspace{0.3cm}
\textbf{Use Case: Run Code}
\begin{itemize}
  \item \textbf{Actor:} Student
  \item \textbf{Description:} The user triggers code execution. The code is sent to the backend compiler.
  \item \textbf{Preconditions:} Valid Haskell code is present in the editor.
  \item \textbf{Postconditions:} Output or errors are returned and displayed.
\end{itemize}

\vspace{0.3cm}
\textbf{Use Case: Chat with AI Assistant}
\begin{itemize}
  \item \textbf{Actor:} Student
  \item \textbf{Description:} The student sends a query about Haskell code or concepts.
  \item \textbf{Preconditions:} An AI session is active.
  \item \textbf{Postconditions:} A helpful, domain-specific response is displayed.
\end{itemize}

\vspace{0.3cm}
\textbf{Use Case: Receive Explanation / Feedback}
\begin{itemize}
  \item \textbf{Actor:} AI Assistant
  \item \textbf{Description:} The assistant analyzes context and provides an explanation, hint, or guidance.
  \item \textbf{Preconditions:} A relevant user query was received.
  \item \textbf{Postconditions:} A response is generated and sent to the user.
\end{itemize}

\vspace{0.3cm}
\textbf{Use Case: Return Output / Errors}
\begin{itemize}
  \item \textbf{Actor:} Compiler Service
  \item \textbf{Description:} Executes code and sends output or error information.
  \item \textbf{Preconditions:} Valid compile request received.
  \item \textbf{Postconditions:} Result is returned to frontend for display.
\end{itemize}

%%%%%%%%%%%%%%%%%%%%%%%%%%%%%%%%%%%%%%%%%%%%%%%%%%%%%%%%%%%%%%%%%%%%%%%%
\newpage
\section{Technology Stack}
%%%%%%%%%%%%%%%%%%%%%%%%%%%%%%%%%%%%%%%%%%%%%%%%%%%%%%%%%%%%%%%%%%%%%%%%

Haskify is implemented using modern web technologies combined with cloud-based AI and compiler integration. The system is structured into a frontend and backend architecture to ensure scalability and maintainability.

\subsection{Frontend}

\begin{itemize}
  \item \textbf{React.js:} Used to build the user interface with component-based design.
  \item \textbf{Vite:} A fast development server and build tool for modern JavaScript applications.
  \item \textbf{HTML/CSS:} Used for layout and styling, including responsive design.
  \item \textbf{Monaco Editor:} A powerful, lightweight code editor (used in Visual Studio Code) embedded into the web application to support Haskell syntax, error highlighting, and real-time editing capabilities.
\end{itemize}

\subsection{Backend}

\begin{itemize}
  \item \textbf{Node.js:} JavaScript runtime used to build the backend server and API routes.
  \item \textbf{Express.js:} Backend web framework for handling API requests.
  \item \textbf{Nodemailer:} Handles email functionality (e.g., contact form).
  \item \textbf{Haskell (GHCi):} Used for compiling and running submitted Haskell code in a sandboxed environment.
\end{itemize}

\subsection{AI Assistant}

\begin{itemize}
  \item \textbf{OpenAI GPT API:} Powers the AI assistant with a specialized focus on functional programming guidance.
  \item \textbf{Custom prompt engineering:} Ensures AI answers remain educational and within domain scope.
\end{itemize}

\subsection{Other Tools and Services}

\begin{itemize}
  \item \textbf{GitHub Pages:} Used for deployment of the frontend.
  \item \textbf{GitHub:} Version control and collaborative code management.
\end{itemize}
