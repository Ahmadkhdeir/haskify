%%%%%%%%%%%%%%%%%%%%%%%%%%%%%%%%%%%%%%%%%%%%%%%%%%%%%%%%%%%%%%%%%%%%%%%%
\chapter{Introduction}
%%%%%%%%%%%%%%%%%%%%%%%%%%%%%%%%%%%%%%%%%%%%%%%%%%%%%%%%%%%%%%%%%%%%%%%%

\section{Why Functional Programming?}

Many universities opt for functional programming as the introductory course for first-year computer science students due to several compelling reasons. Functional programming provides an excellent foundation in algorithms, as it allows algorithms to be expressed clearly and concisely, avoiding unnecessary syntax that can often complicate understanding and implementation~\cite{joosten1993teaching}. Another significant advantage of introducing functional programming early in the curriculum is its emphasis on algorithmic design rather than syntactic details. While imperative programming continues to dominate the field of computer science, teaching alternative programming paradigms such as functional programming is beneficial to broaden students' perspectives. Additionally, many professors and institutions favor functional programming for beginners because it naturally introduces students to essential areas like algebra, artificial intelligence, formal language theory, and specification, making it an effective starting point for these critical topics. Furthermore, it's important to acknowledge that Haskell is one example of a functional programming language.

%%%%%%%%%%%%%%%%%%%%%%%%%%%%%%%%%%%%%%%%%%%%%%%%%%%%%%%%%%%%%%%%%%%%%%%%
\section{Challenges with Learning Functional Programming}

Now, let's discuss some common problems students face while learning programming in general. Many students struggle with understanding abstract concepts, such as how loops function or how methods execute during program runtime~\cite{derus2012difficulties}. Another significant issue is students' difficulty comprehending programming structures and the process of building a complete program. Additionally, mastering the syntax of programming languages is another frequent challenge. Merely understanding pseudo-code is insufficient; students must also grasp specific syntax rules, which can become especially confusing when learning multiple languages simultaneously. Hence, beyond conceptual understanding, practical application and experimentation with code are critical.
Another difficulty students encounter is the lack of sufficient examples provided during lectures. Although this may be intentional, as instructors often encourage independent research, it can be particularly confusing for first-year students who might require more structured guidance. Additionally, available examples sometimes fail to address all possible cases discussed in class. Furthermore, students frequently face challenges with complex data structures~\cite{milne2002difficulties}. This complexity often prompts numerous questions that require immediate clarification. However, tutors or professors are rarely available at all times, and providing instantaneous, real-time answers is practically challenging.
%%%%%%%%%%%%%%%%%%%%%%%%%%%%%%%%%%%%%%%%%%%%%%%%%%%%%%%%%%%%%%%%%%%%%%%%

\section{Proposed Solutions}

Here comes Haskify—but what exactly is Haskify? How does it help, and what does it provide? Haskify is a web application featuring a Haskell code editor with syntax highlighting and an integrated real-time compiler. More importantly, it includes an AI assistant specifically trained for functional programming. Unlike typical general-purpose AI tools such as ChatGPT, Haskify's AI is highly specialized in Haskell and functional programming, providing real-time, around-the-clock support.
This AI assistant has been carefully trained to focus exclusively on functional programming topics, particularly Haskell, ensuring that unrelated queries are not addressed. Even for relevant queries, the AI emphasizes detailed explanations and guidance rather than providing direct solutions immediately. This approach mirrors the role of an experienced university tutor, promoting deeper learning and understanding.
Therefore, Haskify serves as an essential tool for anyone aiming to learn Haskell and functional programming effectively. Users can write code directly in the editor, where the AI assistant promptly identifies and explains errors—be they syntax, logical, or conceptual. Crucially, the assistant guides users through their mistakes by offering insightful hints and constructive explanations without simply handing over the answers, preserving the joy and fulfillment inherent in the learning process.


%%%%%%%%%%%%%%%%%%%%%%%%%%%%%%%%%%%%%%%%%%%%%%%%%%%%%%%%%%%%%%%%%%%%%%%%
